\documentclass[]{article}

%opening
\title{Notes on SQL}
\author{Will Steinhardt}

\begin{document}

\maketitle

\begin{abstract}
These notes will summarize the basics of relational databases and SQL.
\end{abstract}

\section{Glossary of terms}
\begin{itemize}
	\item relational database - a table of data organized into columns and rows.
	\item DBMS - database management system - software for managing a database.
	\item RDBMS - set of software that controls data, i.e. access, organization, and storage.  Includes SQL, Oracle Database, DB2 Warehouse, etc.
	\item DML statements- Data Manipulation Language statements - used to read and modify data. 
	\item predicate - evaluates to true, false, or unknown (as needed in the WHERE clause)
\end{itemize}

\section{Introduction to SQL}
SQL stands for "Structured Query Language" and is an example of a RDBMS. 

Some simple tasks that can be accomplished with SQL are the creation of a table, insertion of data in a table, deletion of data from a table, selection of data from a table, and updating of data in a table.

\subsection{SELECT command}
This command is an example of a query used to retrieve data by selecting columns in a table.

In the simplest example, we can get an entire table with the command:

\begin{verbatim}
SELECT * FROM <table name>;
\end{verbatim}

We can restrict output by specifying the columns 
\begin{verbatim}
SELECT <column_1,  ..., column_n>  FROM <table_name>;
\end{verbatim}

and can further constrain results
using the WHERE clause and an accompanying predicate, i.e.:

\begin{verbatim}
SELECT SELECT <column_1,  ..., column_n>  FROM <table_name> WHERE <predicate>;
\end{verbatim}


\subsection{COUNT function}

In order to count the number of entries in a table you can use the following command:

\begin{verbatim}
SELECT COUNT(*) FROM <table name>;
\end{verbatim}

By the same token, if you wanted to count according to a particular criterion, you could do so in the following way:

\begin{verbatim}
SELECT COUNT(<column name>) WHERE <predicate>;
\end{verbatim}

\subsection{DISTINCT}

In order to remove duplicate values from a result set, use the \verb DISTINCT  statement:

\begin{verbatim}
SELECT DISTINCT <column name> FROM <table name> WHERE <predicate>;
\end{verbatim}

\subsection{LIMIT}

You can restrict the output set to the first \emph{n} results by appending your selection in the following way:

\begin{verbatim}
SELECT * FROM <table name> LIMIT <n>;
\end{verbatim}

Of course, this can also be used in conjunction with other statements:

\begin{verbatim}
SELECT <column name> FROM <table name> WHERE <predicate> LIMIT <n>;
\end{verbatim}

/subsection{INSERT}

This is another example of a DML statement, used for inserting rows into a table.

\subsection{INSERT}

We can add new rows to the table by using the \verb INSERT  statement:

\begin{verbatim}
INSERT INTO  <table name> (<column name 1, ... column name n>) 
	VALUES (<value 1, ..., value n>);
\end{verbatim}

If multiple sets of values are included in parentheses separated by commas, multiple rows may be added simultaneously.

\subsection{UPDATE}

To modify the value of a column, one can use the following command:

\begin{verbatim}
UPDATE <table name> SET <column name> = <value> WHERE <predicate>;
\end{verbatim}

Here the use of the \verb WHERE   clause is particularly important, because you likely do not want to change all rows of a table.

\subsection{DELETE}

\verb DELETE  can be used to remove rows from a table using the following syntax:

\begin{verbatim}
DELETE FROM <table name> WHERE <predicate>;
\end{verbatim}

Note that here the predicate might take the form:

\begin{verbatim}
<column name> IN (<value 1>, ..., <value n>)
\end{verbatim}

\verb IN  here is used to specify every value found \emph{in} a particular set of values.

\section{Refining and Sorting Result Sets}

\subsection{Searching with more general criteria}
If you want to find data by matching more general criteria, you can modify the predicate of \verb|WHERE|.

\begin{verbatim}
WHERE <column name> LIKE 'R%';
\end{verbatim}

Here \% is a wildcard, and if the column selected has strings for the data type, those beginning with `R' will be returned.

By the same token, you can specify numerical ranges for data in the following way:

\begin{verbatim}
SELECT <column name 1>, <column name 2> FROM <table name> WHERE
 <column name 2> BETWEEN <lower number> AND <higher number>;
\end{verbatim}

\subsection{Ordering Result Sets}

Use \verb|ORDER BY| to order a result set.

\begin{verbatim}
SELECT <column name> FROM <table name> ORDER BY <column name>;
\end{verbatim}

The keyword \verb|DESC| can be added at the end to switch the order to descending.

If the column is numerical, you can specify \verb|ORDER BY <column number (in query)>| and the result set will be sorted according to the specified column's value.  Multiple ordering conditions can be included in sequence.

\subsection{DISTINCT}
If you want to eliminate redundant results, you can use \verb|DISTINCT|.

\begin{verbatim}
SELECT DISTINCT(<column name>) FROM <table name>;
\end{verbatim}

\subsection{GROUP BY}

Result sets can be grouped using \verb|GROUP BY|.  For example the following query will first group the table according to similar <column name> entries, and then count the output for each (providing two columns, the distinct entries and number for each):

\begin{verbatim}
SELECT <column name>, count(<column name>) FROM <table name> 
GROUP BY <column name>;
\end{verbatim}

To add a meaningful column label, we can add \verb|AS Count| after the \verb|count()| command to make the result set easier to understand.

The result set can be further refined by adding \verb|HAVING <predicate>|:

\begin{verbatim}
SELECT <column name>, count(<column name>) FROM <table name> 
GROUP BY <column name> HAVING <predicate>;
\end{verbatim}

\subsection{Specifying tables with AS}

One can query multiple tables by denoting each table with a specified label (usually a letter) and the \verb|AS| clause. 

\begin{verbatim}
SELECT A.<column name 1>, B.<column name 2> FROM <table name 1> 
AS A, <table name 2> AS B;
\end{verbatim}

\section{Built-in functions}
Built-in functions can be used to get information from a database directly.  User-defined functions can be made.

\subsection{Aggregate functions}
\verb|AVG()|,
\verb|MIN()|,
\verb|MAX()|,
\verb|SUM()|.


\subsection{Scalar and string functions}
\verb|ROUND()|,
\verb|LENGTH()|,
\verb|UCASE()|,
\verb|LCASE()|.

These can be used in the \verb|WHERE| clause, which can be useful if you are not sure whether data is upper or lower case, for example.

Functions can be nested - one can be the input for another.

\subsection{Date and time}
\verb|DAY()|,
\verb|MONTH()|,
\verb|YEAR()|.

You can perform arithmetic using dates - for example: 
\begin{verbatim}
SELECT <time column> + 3 DAYS FROM <table name>;
\end{verbatim}

You can also use the special registers \verb|CURRENT_DATE| and \verb|CURRENT_TIME|.

\section{Sub-queries}
Built-in functions cannot always be evaluated in a predicate. To circumvent this, it is often useful to include an entire \verb|SELECT| statement in parentheses where one might otherwise want to use a function.  As an example:

\begin{verbatim}
SELECT <column name> FROM <table name> WHERE <column name> < 
(SELECT AVG(column name) < 10);
\end{verbatim}

Another option is column expressions where in place of a column name, a full \verb|SELECT| statement is included in parentheses.

\begin{verbatim}
SELECT <column name>,(SELECT <column name> FROM <table name>) 
FROM <table name>;
\end{verbatim}

Table expressions can be used to define an instance of a table representing a subset of another:

\begin{verbatim}
SELECT * FROM (SELECT <column name 1>, <column name 2> FROM 
<table name>) AS <new table name>; 
FROM <table name>;
\end{verbatim}

\section{Querying multiple tables}

An \emph{implicit join} is when multiple tables are included in the \verb|FROM| clause:

\begin{verbatim}
SELECT * FROM <table name 1>, <table name 2>;
\end{verbatim}

This results in a `full' or `Cartesian' join, where every row from the second table is joined with every row in the first.

When using implicit joins, it is possible to refine the result set.  Columns should accessed with \verb|<table name>.<column name>| to account for the possibility that the same column name is found in each table.

The names of tables can be defined in shorthand:

\begin{verbatim}
SELECT * FROM <table name 1> A, <table name 2> B 
WHERE A.<column name 1> = B.<column name 2>;
\end{verbatim}

\section{Using python to work with SQL databases}

DB API is a standard API for using python to work with databases (not only a particular flavor of SQL).

There are two main concepts in working with a Python DB API:

\emph{Connection object} - used to connect to databases and manage transactions.

\emph{Cursor object} - used for queries, to scan through and retrieve result sets.

As an example of connecting to a DB with python:

\begin{verbatim}
from dbmodule import connect

#Create connection object

Connection = connect('databasename','username','pswd')

#Create a cursor object
Cursor=connection.cursor()

#Run Queries
Cursor.execute('select*from mytable')
Results=cursor.fetchall()

#Free resources
Cursor.close()
Connection.close()
\end{verbatim}

\subsection{Working with ibm\_db and pandas}
Tables can be created using the \verb|ibdm.exec_immediate()| function.  The arguments for the function are: \verb|idm.exec_immediate(conn,"<SQL query>")|.  Of course, here \verb|conn| is a connection object as described above.

You can define a SQL statement and then pass it to another ibm\_db function:
\begin{verbatim}
statement = ibm_db.exec_immediate(conn,"SELECT * FROM table")

ibm_db.fetch_both(statement)
\end{verbatim}

We can use pandas to load dataframes as well.

\begin{verbatim}
import pandas
import ibm_db_dbi

pconn = ibm_db_dbi.Connection(conn)
df = pandas.read_sql('SELECT * FROM table', pconn)
df
\end{verbatim}
\subsection{Magic statements}
SQL queries can be easily entered using ``magic" commands in a Jupyter notebook:

\verb|%load_ext sql|

\verb|%sql select * from tablename|

You can run an entire cell in Jupyter as SQL by adding \verb|%%sql| on the first line.

Furthermore, you can take advantage of variables defined in Python in the scripts:

\begin{verbatim}
country1 = "USA"
%sql select * from <table name> where country = :country1
\end{verbatim}

By the same token, you can assign the results of queries to python variables:

\begin{verbatim}
thing = %sql select * from <table name>
\end{verbatim}

\section{Obtaining table metadata}
In order to get information about the tables in a relational database, one can use:

\begin{verbatim}
SELECT * FROM syscat.tables
\end{verbatim}

For example:
\begin{verbatim}
SELECT TABSCHEMA, TABNAME, CREATE_TIME FROM syscat.tables
 WHERE TABSCHEMA = '<db2 username>'
\end{verbatim}

\end{document}

